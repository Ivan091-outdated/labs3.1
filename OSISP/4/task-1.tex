\newcommand{\basefile}{../../src/ssp_po5/reports/Karnasevich/4/src/main4/}

\paragraph{Цель работы}
приобрести практические навыки в области ООП

\paragraph{Задание 1}
Создать класс Account (счет) с внутренним классом, с помощью объектов которого можно
хранить информацию обо всех операциях со счетом (снятие, платежи, поступления).


\lstinputlisting[language=Java]{\basefile task1/App1.java}
\lstinputlisting[language=Java]{\basefile task1/Account.java}


\paragraph{Задание 2}
Реализовать агрегирование.
При создании класса агрегируемый класс объявляется как атрибут (локальная переменная, параметр метода).
Включить в каждый класс 2-3 метода на выбор.
Продемонстрировать использование разработанных классов.

Создать класс Страница, используя класс Абзац.

\lstinputlisting[language=Java]{\basefile task2/App2.java}
\lstinputlisting[language=Java]{\basefile task2/Page.java}
\lstinputlisting[language=Java]{\basefile task2/Paragraph.java}

\paragraph{Задание 3}
Построить модель программной системы с применением отношений (обобщения, агрегации, ассоциации, реализации)
между классами. Задать атрибуты и методы классов. Реализовать (если необходимо) дополнительные классы.
Продемонстрировать работу разработанной системы.

Система Больница. Пациенту назначается лечащий Врач. Врач может сделать назначение
Пациенту (процедуры, лекарства, операции). Медсестра или другой Врач выполняют на-
значение. Пациент может быть выписан из Больницы по окончании лечения, при нарушении
режима или иных обстоятельствах.

\lstinputlisting[language=Java]{\basefile task3/Main.java}
\lstinputlisting[language=Java]{\basefile task3/Clinic.java}
\lstinputlisting[language=Java]{\basefile task3/Doctor.java}
\lstinputlisting[language=Java]{\basefile task3/Medic.java}
\lstinputlisting[language=Java]{\basefile task3/Patient.java}

