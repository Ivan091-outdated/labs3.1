\paragraph{Вывод:}
Семафор представляет собой atomic unsigned int.
По сути семафор --- не бинарный мьютекс.
При помощи семафора можно синхронизировать работу нескольких процессов.
В Linux существуют именованные и неименованные семафоры.
Для выполнения задания я выбрал именно неименованную вариацию, так как это позволяет получить к такому семафору доступ из любого процесса.
Операции с семафорами:
\begin{itemize}
    \item \texttt{sem\_open()} --- создание семафора или получение доступа на уже существующий.
    Также в эту функцию может входить инициализация семафора.
    \item \texttt{sem\_unlink()} --- удаление именованного семафора.
    \item \texttt{sem\_wait()} --- уменьшение значения семафора.
    Если оно уже равно нулю, то процесс останавливается до тех пор, пока значение не увеличится.
    \item \texttt{sem\_post()} --- увеличение значения семафора.
    Если существует остановленный этим семафором процесс, то он возобновляется.
\end{itemize}