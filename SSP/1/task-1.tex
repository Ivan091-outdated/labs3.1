\newcommand{\baseFile}{../../src/ssp_po5/reports/Karnasevich/1/src/main1/Application.java}

\paragraph{Цель работы}
приобрести практические навыки обработки параметров командной строки, закрепить базовые
знания языка программирования Java при решении практических задач.

\paragraph{Задание 1}
Вывод моды последовательности. Модой ряда чисел называется число, которое встречается в
данном ряду чаще других. Последовательность может иметь более одной моды, а может не
иметь ни одной.

\lstinputlisting[language=Java, firstline=17, lastline=32]{../../src/ssp_po5/reports/Karnasevich/1/src/main1/Application.java}

\paragraph{Задание 2}
Написать метод shiftRight(double[] array, int shift), который сдвигает элементы массива
array на заданное число позиций shift вправо.

\lstinputlisting[language=Java, firstline=34, lastline=43]{../../src/ssp_po5/reports/Karnasevich/1/src/main1/Application.java}

\paragraph{Задание 3}
Напишите метод boolean pangramEng(String str), проверяющий, является ли строка панграммой или нет.
Панграмма – это такая строка, которая содержит все или почти все буквы алфавита, по возможности не повторяя их.

\lstinputlisting[language=Java, firstline=45, lastline=51]{../../src/ssp_po5/reports/Karnasevich/1/src/main1/Application.java}