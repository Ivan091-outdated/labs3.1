\paragraph{Цель работы}
приобрести базовые навыки работы с файловой системой в Java

\paragraph{Задание 1}
Напишите программу, выполняющую чтение текстовых данных из файла и их последующую обработку:

Напишите программу выдачи перекрестных ссылок, т.е. программу, которая выводит список
всех слов документа и для каждого из этих слов печатает список номеров строк, в которые это
слово входит.

\lstinputlisting[language=Java]{../../src/ssp_po5/reports/Karnasevich/2/src/main2/App1.java}

\paragraph{Задание 2}
Утилита tail выводит несколько (по умолчанию 10) последних строк из файла.
Формат использования: tail [-n] file

Ключ -n <количество строк> (или просто <количество строк> ) позволяет изменить количество выводимых строк.

Пример использования:

\texttt{tail -n 20 app.log}

\texttt{tail 20 app.log}

Выводит 20 последних строк из файла app.log.

Для решения задачи подойдет класс java.io.RandomAccessFile, реализующий произвольный доступ к файлу (чтение и запись с любой позиции в файле).

\lstinputlisting[language=Java]{../../src/ssp_po5/reports/Karnasevich/2/src/main2/App2.java}

\lstinputlisting[]{../../src/ssp_po5/reports/Karnasevich/2/src/head.bat}