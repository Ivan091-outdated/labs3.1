\subparagraph{Задание 2:}
Вычислить значения выражений.
\begin{ttfamily}
    \begin{enumerate}
        \item m+++n
        \item m-{}- > n
        \item n-{}- > m
    \end{enumerate}
\end{ttfamily}
Объяснить полученные результаты.

\lstinputlisting[label={lst:cpptask2}, language=C++]{../src/cpp/lab1/task2.cpp}
\begin{lstlisting}[label={lst:bash}]
ivan@pc:~/Labs/latex/src/cpp/lab1$ g++ task2.cpp -o t2
ivan@pc:~/Labs/latex/src/cpp/lab1$ ./t2
7
0
1
\end{lstlisting}
\begin{enumerate}
    \item \ttfamily{m+++n} на самом деле \ttfamily{( m++ ) + n}
    \item \ttfamily{m-{}- > n} декремент постфиксный, потому сначала вычислится
            \ttfamily{m} и вставится в выражение, а потом \ttfamily{m} уменьшится на 1.
    \item \ttfamily{n-{}- > m} см.{} предыдущий пункт.
\end{enumerate}