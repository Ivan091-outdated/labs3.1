\paragraph{Часть 1}
\subparagraph{Задание 1}

Определить путевое имя рабочего каталога. Как обозначается корневой
каталог? Какое путевое имя получили (относительное или абсолютное)?

\begin{verbatim}
ivan@pc:~$ pwd
/home/ivan
\end{verbatim}

Получен абсолютный путь.

Корневой каталог обозначается \textbf{\textbackslash}.

\subparagraph{Задание 2}

Создать в начальном каталоге два подкаталога. Просмотреть содержимое
рабочего каталога. Просмотреть содержимое родительского каталога, не
переходя в него.

\begin{verbatim}
ivan@pc:~/lab$ mkdir dir1
ivan@pc:~/lab$ mkdir dir2
ivan@pc:~/lab$ ls
dir1  dir2
ivan@pc:~/lab$ ls ../
books          Downloads     Music            snap
CLionProjects  external      Pictures         Templates
configuration  IdeaProjects  Programs         test
Desktop        lab           Public           Videos
Documents      Latex         PycharmProjects
\end{verbatim}

\subparagraph{Задание 3}

Перейти в системный каталог. Просмотреть его содержимое. Просмотреть
содержимое начального каталога. Вернуться в начальный каталог.

\begin{verbatim}
ivan@pc:~/lab$ cd /
ivan@pc:/$ ls
bin    dev   lib    libx32      mnt   root  snap  tmp
boot   etc   lib32  lost+found  opt   run   srv   usr
cdrom  home  lib64  media       proc  sbin  sys   var
ivan@pc:/$ ls ~lab
ls: cannot access '~lab': No such file or directory
ivan@pc:/$ ls ~/lab
dir1  dir2
ivan@pc:/$ cd ~/lab
\end{verbatim}

\subparagraph{Задание 4}

Удалить созданные ранее подкаталоги.

\begin{verbatim}
ivan@pc:~/lab$ rm -r dir1
ivan@pc:~/lab$ rm -r dir2    
\end{verbatim}

\subparagraph{Задание 5}

Получить информацию по командам ls и cd с помощью утилиты man. Изучить
структуру man-документа.

\begin{verbatim}
ivan@pc:~/lab$ man ls
ivan@pc:~/lab$ man cd
No manual entry for cd
\end{verbatim}

man даёт информацию о команде.

Структура man:
\begin{enumerate}
    \item имя команды и краткое её описание
    \item паттерн использования команды
    \item подробное описание команды и аргументов командной строки
    \item автор
    \item ссылки, куда репортить баги
    \item копирайт
    \item дополнительная информация
\end{enumerate}

\subparagraph{Задание 6}

Получить краткую информацию по командам ls и cd с помощью команды
whatis и apropos. В чем различие?

whatis даёт краткую информацию по конкретной команде, 
а apropos ищет вхождения своего аргумента 
в командах и даёт информацию по ним.

\subparagraph{Задание 7}

То же, что и в п.5, только с помощью команды info.

info даёт более детальную информацию в отличии от man. Эта информация представлена
в другом формате.