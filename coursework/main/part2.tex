\section{Проектирование системы}\label{sec:part2}
Система в целом должна состоять из textbf{клиента}, textbf{сервера} и textbf{БД}.
Общая схема отображена на рисунке \ref{fig:mainArchitecture}.

\begin{figure}[ht]
    \centering
    \includegraphics[width=0.9\textwidth]{../resources/mainSystem.png}
    \caption{Общая архитектура приложения.}
    \label{fig:mainArchitecture}
\end{figure}

\subsection{Архитектура БД}
Современные программные системы редко обходятся без энергонезависимого хранилища данных.
Это могут быть реляционные, нереляционные (NoSQL), графовые БД.
В конце концов данные можно хранить в простых текстовых файлах.

В разрабатываемой системе используется реляционная БД.
Это самый простой, хорошо изученный и понятный способ хранить данные.
В контексте этого приложения нет смысла использовать NoSQL решения, основные преимущества которых, а именно высокая гибкость, попросту бессмысленны в данной системе.

\begin{figure}[ht]
    \centering
    \includegraphics[width=0.9\textwidth]{../resources/schemaSQL.png}
    \caption{Общая архитектура приложения.}
    \label{fig:mainDBArchitecture}
\end{figure}

Как видно на рисунке \ref{fig:mainDBArchitecture}, основные сущности: STUDENT, LESSON, TEACHER, STUDENT\_GROUP.
\begin{itemize}
    \item STUDENT\_GROUP позволяет создавать группы студентов, при этом STUDENT и STUDENT\_GROUP связаны отношением Many-to-many, что позволяет удобно назначать занятие сразу множеству студентов.
    \item STUDENT\_APPOINTMENT хранит статус посещения занятия студентом.
    \item Каждый LESSON связан с PERIOD. PERIOD здесь выступает номером занятия (1-ая, 2-ая пара и т.д.)
\end{itemize}

В качестве конкретной БД использована H2, как простое in-memory решение, что сильно ускоряет разработку.
В боевом применении лучше использовать H2 в Server или Mixed режиме, также хорошим выбором будет PostgreSQL.

\subsection{Архитектура сервера}
Серверная сторона должна отвечать следующим критериям:

\begin{itemize}
    \item Достаточные познания меня, как разработчика, в выбранной платформе.
    \item Возможность легко взаимодействовать с реляционной БД
    \item Удобные инструменты реализации REST API
    \item Поддержка Server Sent Events (SSE)
    \item Доступные инструменты разработки
    \item Простота включения внешних библиотек в приложение
\end{itemize}

К критериям выше идеально подходит Java-Spring \cite{SpringReference} платформа.
Общая схема сервера на Spring выглядит следующим образом:

\begin{figure}[ht]
    \centering
    \includegraphics[width=0.6\textwidth]{../resources/serverArchitecture.png}
    \caption{Общая архитектура сервера.}
    \label{fig:mainServerArchitecture}
\end{figure}

\begin{itemize}
    \item Controller формирует REST API, также отвечает за создание SSE.
          При получении HTTP запроса, в этом модуле формируются соответствующие Data Transfer Object (DTO),
          поля DTO валидируются, после управление передается в Service.
    \item В Service располагается логика приложения выполняет.
          Он обращается к Repository по необходимости. Также инфраструктурные задачи лежат в основном на Service.
    \item Repository отвечает на взаимодействие с БД.
          В этом слое объекты данные, превращаются в SQL запросы, а результаты этих запросов --- в объекты Model.
\end{itemize}

\subsubsection{Repository}
Для взаимодействия с БД в Java обычно используют JPA, в частности Hibernate \cite{HibernateReference}.
В Spring даже есть абстракция над JPA --- Spring Data JPA \cite{SpringDataJPAReference}.
Hibernate, как проверенный инструмент выбран в качестве инструмента работы с БД.


\subsubsection{Build tools}
Сборка современного Java приложение уже давно перестала быть тривиальной задачей.
Причина тому --- невероятное количество используемых библиотек.
Для облегчения сборки специальные инструменты, такие как: Apache Ant, Apache Maven, Gradle

\begin{itemize}
    \item Apache Ant был одним из первых инструментов автоматической сборки проектов для Java.
          Ant --- императивный инструмент, поэтому каждое действие должно быть описано в файле конфигурации (в данном случае XML).
          Это придаёт невероятную гибкость во время сборки, но в тоже время, Ant конфигурация может быть невероятно длинной и сложной.
    \item Apache Maven в отличие от Ant --- полностью декларативный инструмент.
          Начать новый проект на Maven очень просто, ведь он сам знает, как и куда загружать зависимости проекта, где находится XML конфигурация самого Maven и т.д.
          От программиста требуется всего-лишь следовать его ожиданиям.
          Слабая сторона Maven --- кастомизация.
          Для задания дополнительной логики необходимо использовать Maven-плагины, которые не всегда полностью покрывают требования.
    \item Gradle находит баланс между императивностью Ant и декларативностью Maven.
          В Gradle отказались от XML заменив его Domain Specific Language (DSL).
          DSL позволяет декларативно описать сборку на Maven, а если нужно, то написать живой код на Groovy или Kotlin (каждый из которых является JVM языком).
\end{itemize}

Для проекта выбран Gradle, как одновременно простой, но гибкий инструмент.
Gradle также предоставляет разбить цельное приложение на модули.

\begin{figure}[ht]
    \centering
    \includegraphics[width=0.6\textwidth]{../resources/moduleDependencies.png}
    \caption{Зависимости модулей друг от друга.}
    \label{fig:moduleDependencies}
\end{figure}

\subsection{Архитектура клиента}
На клиенте жизнь, в отличие от сервера, сильно проще.

Здесь уже есть стандартный пакетный менеджер npm.

Между Angular, View и React я выберу React, так как из вышеперечисленного я знаю только его.

В качестве языка программирования выбран JavaScript.
Можно было выбрать TypeScript, но для небольшого проекта он породит больше проблем, чем решит.

Централизованное управление состоянием, такое как Redux или MobX, использовать вполне нужно.
Для этой цели выбран Redux.

Для сканирования QR кода будет использована библиотека \texttt{react-qr-reader}, для его генерации --- \texttt{qrcode.react}.

