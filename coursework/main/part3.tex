\section{Реализация системы}\label{sec:devAlgoParallel}

\subsection{Инструменты}

Среда разработки --- Intellij IDEA (Для сервера и клиента).

Сервер бэка --- Embedded apache Tomcat.

Сервер фронта --- Node.js.

Для написания пояснительной записки использован VS Code и Latex.

\subsection{Сервер}

На стороне сервера используется Gradle в качестве системы сборки.
Однако, для запуска сервера отдельно устанавливать и настраивать окружение Gradle не нужно.
Gradle-wrapper.jar идёт в комплекте с исходным кодом, что значительно упрощает развёртывание.
Конфигурация Gradle в проекте представляет собой код на языке Kotlin.

\subsubsection{Контроллеры}

Каждый Rest контроллер --- Spring bean, у которого методы помечены @RequestMapping (Может быть put, get и т.п.)
Всего в проекте 3 контроллера:

\begin{enumerate}
    \item CheckinController --- отвечает за чекин студентов.
    \item UserController --- отвечает за авторизацию.
    \item LessonMonitoringController --- позволяет получить информацию о текущем занятии.
\end{enumerate}

\subsubsection{SSE}

С проекте используются SSE, который прекрасно поддерживается в Spring.
Всё, что нужно сделать, вернуть из контроллера объект типа SseEmitter.
У этого объекта есть метод send, благодаря которому можно отправить данные на клиент.

Сервер работает с несколькими SSE одновременно.
Поэтому, все SseEmitter-ы хранятся в оперативной памяти.
В качестве коллекции выбран ConcurrentHashMap<Integer, CachingSseEmitter>, где ключ --- id занятия, а значение --- надстройка над обычными SseEmitter, запоминающая предыдущее сообщение, что позволяет использовать одноразовые QR коды.

\subsubsection{Spring Data Jpa}

Spring Data Jpa --- часть стека Spring.
Хотя это всего лишь надстройка над JPA, которая в свою очередь надстройка наж JDBC.
По умолчанию Spring Data JPA использует Hibernate.

Сама концепция Hibernate подразумевает абстрагирование от реляционной БД при помощи построение двусвязного графа сущностей в Java.
Эти сущности --- Java объекты с конструктором по умолчанию, геттером и сеттером к каждому полю.
Также они не могут быть final.
Названия полей должны совпадать с названиями колонок (Hibernate сам умеет переводить camel-case в snake case).
Сущности JPA обильно приправлены аннотациями (@OneToMany, @Entity и проч).
По этим аннотациям Hibernate строит модель БД.

Для работы с БД в Spring Data JPA используется паттерн репозиторий.
Согласно нему сущности не умеют сами себя сохранять в БД, но для этого существуют специальные объекты --- репозитории.
Они могут строить SQL на основе имени метода, HQL, и нативного SQL.

Вся работа с БД сосредоточена в модуле repo.
Тут располагаются все JPA репозитории и сущности.

\subsection{Клиент}

React приложение создано при помощи create-react-app.
Из него убраны все иконки React.
Для оформления использован Matherial ui от Google.

Приложение делится на 3 модуля:
\begin{enumerate}
    \item api --- доступ к эндпоинтам (используется axios).
    \item components --- все React компоненты находятся здесь.
    \item redux --- здесь располагается управление состоянием приложения.
\end{enumerate}

Клиент сам по себе делится на 2 части --- для студента и преподавателя.
Эти 2 части делят между собой форму логина.

В части преподавателя находится SSE части для генерации одноразового QR кода.
В части студента --- модуль сканирования QR кода и отправки хэша на сервер.
При авторизации клиент получает информацию о роли пользователя, так он может выбрать вариант представления.
