\newpage


\section{Разработка и описание последовательного алгоритма программной системы}\label{sec:devalgorythms}
Исходное изображение представляет собой матрицу пикселей, где каждый элемент имеет значение в пределах \textbf{[0, 255]}.
С такой матрицей работать очень удобно.

\subsection{Общая схема алгоритма.}\label{subsec:devalgorythms}
Схема включает в себя все этапы обработки начиная с загрузки изображения в программу и заканчивая сохранением программы на диске.
\begin{enumerate}
    \item Загружаем изображение в матрицу байт
    \item Определяем, будет ли использован последовательный алгоритм или параллельный
    \item Выполняем алгоритм обработки, включающий в себя подавление шумов и улучшение с использованием операций увеличения и уменьшения.
    Эти алгоритмы используют буферную матрицу пикселей для наиболее корректной обработки изображения.
    \item Сохраняем готовое изображение в файловой системе
\end{enumerate}

\subsection{Подавление шумов.}\label{subsec:denoise}
Подавление шумов с использованием метода усреднения значений задача не слишком сложная.
Так как данные представляют собой матрицу пикселей, задача сводится к простому итерированию по матрице (исключая её края) и изменению значения каждого пикселя в зависимости от значений его соседей.
А именно эти значения складываются в переменную аккумулятор, потом значение аккумулятора делится на 8 (эту операцию можно заменить битовым сдвигом на 3 влево) и это финальное значение помещается в буферную матрицу, так как изменение изначальной матрицы во время выполнения алгоритма снижает его точность.
При этом крайние пиксели изображения не обрабатываются сами, но используются для обработки своих соседей.

\subsection{Улучшение изображения используя увеличение и уменьшение.}\label{subsec:upgrade}
Данная операция не столь однозначна, как подавление шумов.
Дело в том, что увеличения у уменьшения можно делать многократно и в разном порядке.
Также эти операции используют пороговые значения.
Поэтому это уж очень индивидуально для каждого изображения.

В целом, уменьшение представляет собой простое итерирование во всем пикселям картинки, где каждый пиксель становится чёрным (его значение становится равно 0), если он имеет хоть одного соседа, значение которого меньше порогового.
Очень важно использовать буферную матрицу при таком преобразовании, результат обработки будет совершенно ужасным, а именно всё изображение рискует стать полностью чёрным.

Увеличение очень похоже на уменьшение, только пиксель становится белым (значение 255), если у него есть сосед, значение которого выше порогового.