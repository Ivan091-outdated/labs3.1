\section{Системный анализ и постановка задачи}\label{sec:part1}
Основной сценарий использования системы:
\begin{enumerate}
    \item Все пользователи регистрируются в системе администратором.
    \item Каждый пользователь имеет роль: преподаватель или студент.
    \item Список занятий заносится в базу данных (БД) администратором.
    \item Во время очередного занятия, преподавать на своём устройстве начинает занятие, у него появляется одноразовый QR-код, который сканируется студентами.
    \item Студент, просканировавший QR-код считается посетившим занятие.
\end{enumerate}

Из задачи, очевидно, что необходимо использовать какое-нибудь энергонезависимое хранилище, иначе получить статистику посещаемости просто невозможно.

Также необходимо написать WEB сервер, реализовать REST API на нём для взаимодействия с клиентом.
Сервер также должен поддерживать WebSockets или Server Sent Events для обновления QR по инициативе сервера.

Клиент стоит сделать максимально простым и тонким.
Для достижения кроссплатформенности максимально простым способом, проще всего сделать именно браузерный клиент.
При разработке клиента стоит придерживаться Mobile First подхода, ввиду того, что большинство сценариев использования (сканирование QR кода, например), чаще всего производится посредством мобильного устройства.

К опциональным задачам стоит отнести:
\begin{itemize}
    \item Вывод статистики по каждому конкретному студенту, преподавателю, предмету и т.п.
    \item Создание отдельного клиента для администратора
\end{itemize}

При реализации основного функционала, опциональные задачи не должны доставить особых проблем, ведь для их решения необходимо всего лишь расширить REST API.