\section*{Введение. Анализ задачи проектирования.}
\addcontentsline{toc}{section}{\protect\numberline{}Введение. Анализ задачи проектирования.}
С появлением многопроцессорных компьютеров параллельное программирование играет важнейшую роль в обработке информации.
Сегодня уже невозможно представить процессор только с одним ядром или сервер работающий в однопоточном режиме.
Поэтому понимание многопоточности --- это полезное умение любого программиста.
Большинство операционных систем, особенно интерактивных, работают сразу на всех ядрах процессора одновременно.
Также ОС предоставляет интерфейс взаимодействия с потоками и процессами, а также управление их жизненным циклом.
В языке C++, как в одном из наиболее низкоуровневых достаточно инструментов работы с потоками:
\begin{enumerate}
    \item MPI позволяет работать не только в пределах одной машины, но и связывать много процессов в единый кластер.
    \item OpenMP предоставляет наиболее простой интерфейс распараллеливания C++ кода, поэтому я предпочитаю использовать его.
    \item Thread родной для С++ модуль, но он слегка многословен, поэтому я решил от него отказаться.{}
\end{enumerate}
На самом деле неважно какие именно инструменты используются при распараллеливании алгоритмов и программ, принципы от этого не меняются.

Задача разработанной программы --- обработка изображений.
Задача сводится к применению нескольких алгоритмов к матрице байт.
Если алгоритм не сложен, то распараллелить его не составит никакого труда.