\newcommand{\basefile}{../../src/ssp_po5/reports/Karnasevich/5/src/main5/}

\paragraph{Цель работы}
приобрести практические навыки в области объектно-ориентированного проектирования


\paragraph{Задание 1}
Реализовать абстрактные классы или интерфейсы, а также наследование и полиморфизм для
следующих классов:

interface Сотрудник ← class Инженер ← class Руководитель.

\lstinputlisting[language=Java]{\basefile task1/Main.java}
\lstinputlisting[language=Java]{\basefile task1/Employee.java}
\lstinputlisting[language=Java]{\basefile task1/Engineer.java}
\lstinputlisting[language=Java]{\basefile task1/Manager.java}

\paragraph{Задание 2}
В следующих заданиях требуется создать суперкласс (абстрактный класс, интерфейс) и определить
общие методы для данного класса. Создать подклассы, в которых добавить специфические
свойства и методы. Часть методов переопределить. Создать массив объектов суперкласса и заполнить
объектами подклассов. Объекты подклассов идентифицировать конструктором по имени или
идентификационному номеру. Использовать объекты подклассов для моделирования реальных
ситуаций и объектов.

Создать суперкласс Музыкальный инструмент и классы Ударный, Струнный, Духовой. Со-
здать массив объектов Оркестр. Осуществить вывод состава оркестра.

\lstinputlisting[language=Java]{\basefile task2/Main.java}
\lstinputlisting[language=Java]{\basefile task2/MusicalInstrument.java}
\lstinputlisting[language=Java]{\basefile task2/PercussionInstrument.java}
\lstinputlisting[language=Java]{\basefile task2/StringedInstrument.java}
\lstinputlisting[language=Java]{\basefile task2/WindInstrument.java}


\paragraph{Задание 3}
В задании 3 ЛР No4, где возможно, заменить объявления суперклассов объявлениями абстрактных
классов или интерфейсов.


\lstinputlisting[language=Java]{\basefile task3/Curable.java}
\lstinputlisting[language=Java]{\basefile task3/Doctor.java}
\lstinputlisting[language=Java]{\basefile task3/Patient.java}


