\subparagraph{Вывести любое сообщение с помощью команды echo перенаправив вывод:\\
- в несуществующий файл с помощью символа >;\\
- в несуществующий файл с помощью символа >{}>;\\
- в существующий файл с помощью символа >;\\
- в существующий файл с помощью символа >{}>;\\
Объяснить результаты.}

\begin{verbatim}
ivan@pc:~/lab$ echo haha > 1
ivan@pc:~/lab$ echo haha >> 2
ivan@pc:~/lab$ cat 1
haha
ivan@pc:~/lab$ echo 123456789 > 1
ivan@pc:~/lab$ cat 1
123456789
ivan@pc:~/lab$ echo qwerty >> 1
ivan@pc:~/lab$ cat 1
123456789
qwerty
\end{verbatim}

\begin{itemize}
    \item > и >{}> аналогичны, если файл не существует
    \item > перезаписывает содержимое файла
    \item >{}> добавляет новые данные в конец файла
\end{itemize}

\subparagraph{Переадресовать стандартный ввод для команды cat в файл.}
\begin{verbatim}
ivan@pc:~/lab$ cat 1
123456789
qwerty
ivan@pc:~/lab$ cat 1 > 2
ivan@pc:~/lab$ cat 2
123456789
qwerty
ivan@pc:~/lab$
\end{verbatim}

\subparagraph{Перенаправить stdout и stderr разными способами.}
\begin{verbatim}
ivan@pc:~/lab$ bash script.sh > 1
stderr
ivan@pc:~/lab$ cat 1
stdout
ivan@pc:~/lab$ bash script.sh 2> 2
stdout
ivan@pc:~/lab$ cat 2
stderr
ivan@pc:~/lab$ bash script.sh > 1
stderr
ivan@pc:~/lab$ cat 1
stdout
ivan@pc:~/lab$ bash script.sh 2> 2
stdout
ivan@pc:~/lab$ cat 2
stderr
ivan@pc:~/lab$ bash script.sh > 4 2>>4
ivan@pc:~/lab$ cat 4
stdout
stderr
\end{verbatim}

stdout имеет номер 1, stderr --- 2

Используя \&> можно перенаправить их оба.

\subparagraph{Вывести третью строку из последних десяти строк отсортированного в
обратном порядке файла /etc/group.}

\begin{verbatim}
ivan@pc:~/lab$ sort -r /etc/group | tail -n 10
crontab:x:105:
colord:x:126:
cdrom:x:24:ivan
bluetooth:x:112:
bin:x:2:
backup:x:34:
avahi:x:121:
avahi-autoipd:x:116:
audio:x:29:pulse
adm:x:4:syslog,ivan
\end{verbatim}

\subparagraph{Подсчитать при помощи конвейера команд количество блочных и
количество символьных устройств ввода-вывода, доступных в системе.}
\begin{verbatim}
ivan@pc:~/lab$ ls -l /dev | grep '^b' | wc -l
39
ivan@pc:~/lab$ ls -l /dev | grep '^c' | wc -l
178
\end{verbatim}

Вызываем \texttt{ls}, после отдаём её результат в \texttt{grep}, который ищет \texttt{'b'} или \texttt{'c'} 
в начале строки, после \texttt{wc -l} считает количество строк.

\subparagraph{Написать скрипт, выводящий на консоль все аргументы командной
строки, переданные данному скрипту. Привести различные варианты запуска
данного скрипта, в том числе без непосредственного вызова интерпретатора в
командной строке.}
\begin{verbatim}
ivan@pc:~/lab$ cat script.sh 
#!/bin/sh
for i in $*; do
	echo $i
done
ivan@pc:~/lab$ bash script.sh a s
a
s
ivan@pc:~/lab$ chmod +x script.sh 
ivan@pc:~/lab$ ./script.sh 
ivan@pc:~/lab$ ./script.sh asd asd
asd
asd
\end{verbatim}

\subparagraph{Реализовать командный файл, реализующий меню из трех пунктов (в
цикле):\\
1) ввести пользователя и вывести на экран все процессы, запущенные
данным пользователем;\\
2) показать всех пользователей, в настоящий момент, находящихся в
системе;\\
3) завершение.}
\begin{Verbatim}[breaklines=true, breakanywhere=true]
ivan@pc:~/lab$ cat script2.sh 
#!/bin/bash
echo -e "Info about system:\n$(uname -a)\n"

echo -e "User running the file:\n$(whoami)\n"

echo -e "Enter user's name:"

while :
do
	read name
	if [[ "$name" ==  "q" ]] 
		then break
	fi
	echo -e "List of bash processed run the user:\n"
	ps -u "$name" | grep bash
done

ivan@pc:~/lab$ bash script2.sh
Info about system:
Linux pc 5.8.0-44-generic #50~20.04.1-Ubuntu SMP Wed Feb 10 21:07:30 UTC 2021 x86_64 x86_64 x86_64 GNU/Linux

User running the file:
ivan

Enter user's name:
ivan
List of bash processed run the user:

   9916 ?        00:00:00 bash
  12294 pts/4    00:00:00 bash
  61194 pts/0    00:00:00 bash
  63553 pts/1    00:00:00 bash
  65098 pts/0    00:00:00 bash
q

\end{Verbatim}